% START OF APPENDIX

\appendix
\chapter{Intermediate Codes} \label{appendix:intermediate_codes}
\begin{figure}[H]
    \centering
    \captionsetup{justification=centering}
    \begin{subfigure}{\textwidth}
        \centering
        \begin{minted}[linenos=true, autogobble, frame=lines, framesep=2mm, fontsize=\small]{c++}
            for(double i = 2; i < 9; i += dT_out){
                t_dum += i;
                double enow = gecis(r_pos, vel, Et, t_dum);
                if( enow > maxE ){
                  maxE = enow;
                  t_opt = i;
                }
                t_dum = t;
            }
        \end{minted}
    \end{subfigure} 
    \\
    \begin{subfigure}{\textwidth}
        \centering
        \begin{minted}[linenos=true, autogobble, frame=lines, framesep=2mm,  fontsize=\small]{c++}
            double gecis(double r_pos, double vel, double Et, double &t){
                for(; r_pos >= -R2 && r_pos <= R2 ; t+=dT){
                    vel = c*sqrt(Et*Et-E0*E0)/Et;
                    double RelBeta  = vel/c;
                    double RelGamma = 1.0 / sqrt(1.0-RelBeta*RelBeta);
                
                    double ef=Eradial(r_pos*1000,t,RFphase*deg_to_rad);
                
                    double acc=ef*1E6*eQMratio/(RelGamma*RelGamma*RelGamma); 
                
                    r_pos = r_pos + vel * dT*ns + 1/2*acc*(dT*ns)*(dT*ns);
                    vel=vel+acc*dT*ns;
                    RelBeta  = vel/c;
                    RelGamma = 1.0 / sqrt(1.0-RelBeta*RelBeta);
                    Et=RelGamma*E0; 
                }
                return Et;
            }
        \end{minted}
    \end{subfigure}
    \caption{$L_{out}$ Optimization For Single $\textbf{e}^-$}
    \label{fig:lout_opt_single_e}
\end{figure}

\begin{figure}[H]
        \centering
        \captionsetup{justification=centering}
        \begin{minted}[linenos=true, autogobble, frame=lines, framesep=2mm, fontsize=\footnotesize]{c++}
            int phase_opt(const vector<double>& Louts, int phase_sweep_range){
                double minrms = 1;
                int opt_phase;
                for(int RFphase = -phase_sweep_range; RFphase <= phase_sweep_range; RFphase++){
                    Bunch bunch1(RFphase);
                    double t1 = 0;
                    bunch1.bunch_gecis_t(t1);
                    bunch1.reset_pos();
            
                    for(int i = 0; i < Louts.size(); i++){
                        bunch1.bunch_gecis_d(Louts[i]);
                        bunch1.reset_pos();
                    }
                        
                    if( bunch1.E_rms() < minrms ){
                        minrms = bunch1.E_rms();
                        opt_phase = RFphase;
                    }
                }
                return opt_phase;
            }
        \end{minted}
    \caption{$\phi_{lag}$ Optimization For Tnitial Bunch Design}
    \label{fig:phlag_opt_n_pass}
\end{figure}

\begin{figure}[H]
    \captionsetup[subfigure]{justification=centering}
    \captionsetup{justification=centering}
    \begin{subfigure}{\textwidth}
        \begin{minted}[linenos=true, autogobble, frame=lines, framesep=2mm, fontsize=\small]{c++}
            double vector3d::operator* (const vector3d& other){
                double dot = 0;
                dot += this->x * other.x;
                dot += this->y * other.y;
                dot += this->z * other.z;
                return dot;
            }
        \end{minted}
    \end{subfigure}

    \begin{subfigure}{\textwidth}
        \begin{minted}[linenos=true, autogobble, frame=lines, framesep=2mm, fontsize=\small]{c++}
            vector3d vector3d::operator% (const vector3d& other){
                double x_ = (this->y * other.z) - (this->z * other.y);
                double y_ = (this->z * other.x) - (this->x * other.z);
                double z_ = (this->x * other.y) - (this->y * other.x);
                vector3d crossed(x_, y_, z_);
                return crossed;
            }
        \end{minted}
    \end{subfigure}
    \caption{* and \% operators of \textit{vector3d} class}
    \label{fig:vector3d_dot_cross_product}
\end{figure}

\chapter{Example Simulation Runs} \label{appendix:example_simulation_runs}

\begin{figure}[H]
    \centering
    \captionsetup{justification=centering}
    \begin{minted}[linenos=true, autogobble, frame=lines, framesep=2mm, fontsize=\scriptsize, style=staroffice]{c++}
        Optimal phase with the least RMS : -5

        Simulation settings : 
        ph = -5 deg, gt = 1 ns, enum = 1000
        dT = 0.001 ns, dT_out = 0.01 ns
        
        For the 1th magnet:
        Optimum out path = 0.81044 m
        Magnet guide = 0.25852 m
        Rho = 0.088477 m
        Drift time of the first electron in the bunch : 7.688 ns
        Drift time of the last electron in the bunch : 7.487 ns
        Max energy = 0.47581 MeV
        RMS = 0.0058165 MeV
        
        For the 2th magnet:
        Optimum out path = 1.0833 m
        Magnet guide = 0.37766 m
        Rho = 0.098898 m
        Drift time of the first electron in the bunch : 5.597 ns
        Drift time of the last electron in the bunch : 5.617 ns
        Max energy = 0.89172 MeV
        RMS = 0.0099018 MeV
        
        For the 3th magnet:
        Optimum out path = 1.1705 m
        Magnet guide = 0.41573 m
        Rho = 0.10223 m
        Drift time of the first electron in the bunch : 5.314 ns
        Drift time of the last electron in the bunch : 5.325 ns
        Max energy = 1.298 MeV
        RMS = 0.013879 MeV

        Electron with the most energy : 623) 1.6999 MeV,	RMS of bunch : 0.017981 MeV
        
        Total steps calculated : 12468052652
        Simulation finished in : 632050015 us     ( 632.1 s )
        
    \end{minted}
\caption{$\phi_{lag}$, $\rho$ \& $L$ optimization at \\$P=12$KW, $R_1=0.188$m, $R_2=0.753$m, $t_g=1$ns, $E_{in}=40$KeV}
\label{fig:lout_opt_1ns_Erms}
\end{figure}

\begin{figure}[H]
    \centering
    \captionsetup{justification=centering}
    \begin{minted}[linenos=true, autogobble, frame=lines, framesep=2mm, fontsize=\scriptsize, style=staroffice]{c++}
        Optimal phase with the least RMS : 0

        Simulation settings : 
        ph = 0 deg, gt = 0.8 ns, enum = 1000
        dT = 0.001 ns, dT_out = 0.01 ns
        
        For the 1th magnet:
        Optimum out path = 0.80787 m
        Magnet guide = 0.2574 m
        Rho = 0.088379 m
        Drift time of the first electron in the bunch : 7.629 ns
        Drift time of the last electron in the bunch : 7.48 ns
        Max energy = 0.47579 MeV
        RMS = 0.0038689 MeV
        
        For the 2th magnet:
        Optimum out path = 1.0833 m
        Magnet guide = 0.37765 m
        Rho = 0.098898 m
        Drift time of the first electron in the bunch : 5.589 ns
        Drift time of the last electron in the bunch : 5.605 ns
        Max energy = 0.89169 MeV
        RMS = 0.0068848 MeV
        
        For the 3th magnet:
        Optimum out path = 1.1705 m
        Magnet guide = 0.41573 m
        Rho = 0.10223 m
        Drift time of the first electron in the bunch : 5.311 ns
        Drift time of the last electron in the bunch : 5.318 ns
        Max energy = 1.298 MeV
        RMS = 0.0096887 MeV
        
        Electron with the most energy : 629) 1.6999 MeV,	RMS of bunch : 0.012318 MeV
        
        Total steps calculated : 12455378454
        Simulation finished in : 631136046 us     ( 631.1 s )        
    \end{minted}
\caption{$\phi_{lag}$ \& $\rho$ \& $L$ optimization at \\$P=12$KW, $R_1=0.188$m, $R_2=0.753$m, $t_g=0.8$ns, $E_{in}=40$KeV}
\label{fig:lout_opt_08ns_Erms}
\end{figure}


% END OF APPENDIX

% START OF BIBLIOGRAPHY

\begin{thebibliography}{9}

    \bibitem{rhodo_pottier}
    Jacques Pottier,
    \emph{A new type of rf electron accelerator: The rhodotron},
    Nuclear Instruments and Methods in Physics Research Section B: Beam Interactions with Materials and Atoms,
    Volumes 40–41, Part 2,
    1989,
    Pages 943-945,

    \bibitem{rhodo_prototype}
    J.M. Bassaler, J.M. Capdevila, O. Gal, F. Lainé, A. Nguyen, J.P. Nicolaï, K. Umiastowski,
    \emph{Rhodotron: an accelerator for industrial irradiation},
    Nuclear Instruments and Methods in Physics Research Section B: Beam Interactions with Materials and Atoms,
    Volume 68, Issues 1–4,
    1992,
    Pages 92-95,A:

    \bibitem{rhodos}
    Y. Jongen. (2001). \emph{Manufacturing of Electron Accelerators}. Ion Beam Applications s.a. (IBA) Chemin du Cyclotron 3, B-1348 Louvain-la-Neuve, Belgium.

    \bibitem{cpu_instruction_speed}
    Fukushima, Toshio. (2001). \emph{Reduction of Round-off Errors in Symplectic Integrators}. The Astronomical Journal. 121. 1768-1775. 

    \bibitem{cite:rhodo_design}
    W.~Kleeven, M.~Abs, J.~Brison, E.~Forton, J.~Hubert and J.~Van de Walle,
    \emph{Design and Simulation Tools for the High-Intensity Industrial Rhodotron Electron Accelerator},
    9th International Particle Accelerator Conference,
    2018
    
    
\end{thebibliography}

% END OF BIBLIOGRAPHY

\end{document}
\documentclass[a4paper,oneside,12pt]{report}
\usepackage[utf8x]{inputenc} % To use Unicode (e.g. Turkish) characters
\usepackage{amsmath, amsthm} % Some extra symbols
%\usepackage{subcaption}
\usepackage{styles/fbe_tez}
\usepackage[bottom]{footmisc}
\usepackage{cite}
\usepackage{graphicx}
\usepackage{longtable}
\usepackage{float}
\usepackage{multirow}
\usepackage{algorithm}
\usepackage{algorithmic}
\usepackage{array}
\usepackage{amssymb,bm,cite,graphicx, fixmath, texdraw}
\usepackage{epsfig}
\usepackage{epstopdf}
\usepackage{amsmath}
\usepackage{notoccite}
\usepackage{subcaption}
\usepackage{tabu}
\usepackage{minted}
\usepackage{xcolor}

\usemintedstyle{xcode}
\definecolor{bg}{rgb}{0.95,0.95,0.95}


\interdisplaylinepenalty=2500
\hyphenation{lists} \makeatletter

\newcommand{\field}[1]{\mathbb{#1} }
\newcommand{\beq}{\begin{equation} \setlength\abovedisplayskip{5pt} 
\setlength\belowdisplayskip{5pt}}
\newcommand{\eeq}{\end{equation}}
\newcommand{\bea}{\begin{eqnarray}}
\newcommand{\eea}{\end{eqnarray}}
\newcommand{\defn}{\stackrel{\triangle}{=}}
\newcommand{\nn}{\nonumber}
\newcommand{\nnl}{\nonumber \\}
\renewcommand{\theequation}{\arabic{equation}}
\renewcommand{\labelenumi}{(\roman{enumi})}

\DeclareMathOperator*{\argmin}{arg\,min}
\DeclareMathOperator*{\argmax}{arg\,max}

\def\ifundefined{\@ifundefined}
\def\bfat{\left[ \begin{array}}
\def\emat{\end{array} \right]}
\def\bfatt{\left\{ \begin{array}}
\def\ematt{\end{array} \right.}
\def\bset{\left\{ \begin{array}}
\def\eset{\end{array} \right\}}
\def\bpar{\left( \begin{array}}
\def\epar{\end{array} \right)}

\graphicspath{{figures/}} % Graphics will be here

\newtheorem{thm}{Theorem}[chapter]
\newtheorem{prop}[thm]{Proposition}
\newtheorem{lem}[thm]{Lemma}
\newtheorem{cor}[thm]{Corollary}

\numberwithin{equation}{chapter}


% COVER PAGE
\title{DESING AND SIMULATION SOFTWARE FOR RHODOTRON TYPE ELECTRON ACCELERATORS}
\turkcebaslik{RHODOTRON TİPİ ELEKTRON HIZLANDIRICILARI İÇİN TASARIM VE SİMÜLASYON YAZILIMI}
\degree{B.S., Physics, Boğaziçi University, 2020}
\author{Muhammet Furkan Er}
\program{M.S. Physics}
\subyear{2023}

% APPROVED BY PAGE
\supervisor{Prof. Veysi Erkcan Özcan}
\cosuperi{Assoc. Prof. Gökhan Ünel}
%\cosuperii{Title and Name of Cosupervisor II}
\examineri{Asst. Prof. Bora Akgün}
\examinerii{Assoc. Prof. Bora Işıldak}
%\examineriv{}
%\examinerv{}
\dateofapproval{DD.MM.YYYY}

% BEGIN OF DOCS
\begin{document}
\pagenumbering{roman}
\makemstitle % M.S. thesis
\makeapprovalpage

% ACK PAGE
\begin{acknowledgements}
Acknowledgements come here...
\end{acknowledgements}

% ABS PAGE
\begin{abstract}
    The focus of this study is to design and optimize rose-pattle-type electron accelerators; a new simulation tool called \textit{Rhodotron Simulation} has been developed for this purpose.
    A coaxial cavity with an operating frequency of $107.5$ MHz is currently being manufactured in the Kandilli Detector Accelerator and Instrumentation Laboratory (KAHVELab).
    This new tool's capabilities will be tested with the cavity and will be used to design complementary bending magnets for reaching $1-5$ MeV beam energy.
\end{abstract}

% OZET PAGE 
\begin{ozet}
Bu çalışma, gül yaprağı tipi elektron hızlandırıcıları tasarlamak ve optimize etmek üzerine odaklanmaktadır. 
Bu amaçla \textit{Rhodotron Simulation} adında yeni bir benzetim uygulaması geliştirilmiş, $107.5$ MHz hedef frekansında çalışmak üzere tasarlanmış bir koaksiyal kavitesinin üretimi, Kandilli Algıç Hızlandırıcı ve Enstrümentasyon Laboratuvarı'nda devam etmektedir.
Yeni benzetim uygulaması, kabiliyetleri üretilen kavite ile test edildikten sonra bükme mıknatısları tasarım ve optimizasyonunda $1-5$ MeV demet enerjisine ulaşmak için kullanılacaktır.
\end{ozet}

% CONTENTS AND LIST OF FIG AND TABLE PAGES
\tableofcontents
\listoffigures 
\listoftables

% LIST OF SYMBOLS PAGE
\begin{symbols}
    % The title will be typeset as "LIST OF SYMBOLS".
    % Use a separate \sym command for each symbols definition.
    % First, Latin symbols in alphabetical order
    \sym{$\vec{a}$}{Acceleration}
    \sym{\textbf{$\vec{B}$}}{Magnetic field}
    \sym{c}{Speed of light in vacuum}
    \sym{E}{Energy}
    \sym{$e^-$}{Electron}
    \sym{$e^- - \vec{E}$}{Electron Electric Field Interaction}
    \sym{$e^- - \vec{B}$}{Electron Magnetic Field Interaction}
    \sym{$e^- - EM$}{Electron Electro-Magnetic Field Interaction}
    \sym{\textbf{$\vec{E}$}}{Electric field}
    \sym{f}{Frequency}
    \sym{$\vec{F}$}{Force}
    \sym{$\vec{p}$}{Momentum}
    \sym{$\vec{r}$}{Position}
    \sym{t}{Time}
    \sym{T}{Period}
    \sym{$\vec{v}$}{Velocity}
    \sym{Z}{Shunt Impedance}
    % 1 EMPTY LINE BETWEEN LATIN AND GREEK SYMBOLS GROUP!!!
    \sym{}{}
    % Then Greek symbols in alphabetical order
    \sym{$\gamma$}{Lorentz Factor}
    \sym{$\beta$}{Beta Factor}
    \sym{$\phi_{lag}$}{Phase Lag}

    \sym{$\mu X$}{Mean Value of X}
    \sym{$\sigma X$}{Standard Deviation of X}
    \sym{}{}
\end{symbols}

% LIST OF ABBREV PAGE
\begin{abbreviations}
    % Abbreviations in alphabetical order
   \sym{2D}{One Dimensional}
   \sym{2D}{Two Dimensional}
   \sym{3D}{Three Dimensional}
   \sym{eV}{Electron Volt}
   \sym{Hz}{Hertz}
   \sym{KAHVELab}{Kandilli Detector Accelerator and \\Instrumentation Laboratory}
   \sym{Linac}{Linear Particle Accelerator}
   \sym{ODE}{Ordinary Differential Equation}
   \sym{RF}{Radio Frequency}
   \sym{RMS}{Root Mean Square}
   \sym{POC}{Proof of Concept}
   \sym{CPU}{Central Processing Unit}
   \sym{s}{Second}
   \sym{T}{Tesla}
   \sym{V}{Volt}
   \sym{W}{Watt}

\end{abbreviations}


\pagenumbering{arabic}



\newcommand{\vecthreeBF}[1]{\vec{\textbf{#1}}}
\newcommand{\vecthree}[1]{\vec{#1}}
\newcommand{\vecNum}[3]{(#1, #2, #3)}

\newcommand{\parDeriv}[2]{\frac{\partial #1}{\partial #2}}
\newcommand{\parDerivS}[2]{\frac{\partial^2 #1}{\partial #2^2}}
\newcommand{\derivS}[2]{\frac{d^2 #1}{d#2^2}}

\newcommand{\dotProdBF}[2]{\vecthreeBF{#1} \cdot \vecthreeBF{#2}}
\newcommand{\dotProd}[2]{\vecthree{#1} \cdot \vecthree{#2}}

\newcommand{\crossProdBF}[2]{\vecthreeBF{#1} \times \vecthreeBF{#2}}
\newcommand{\crossProd}[2]{\vecthree{#1} \times \vecthree{#2}}

\newcommand{\e}{$\textbf{e}^-$ }
\newcommand{\egun}{$\textbf{e}^-$-gun }
\newcommand{\eB}{$\textbf{e}^-$ - $\vecthreeBF{B}$ }
\newcommand{\eE}{$\textbf{e}^-$ - $\vecthreeBF{E}$ }
\newcommand{\eEM}{$\textbf{e}^-$ - \textbf{EM} }
\newcommand{\ee}{$\textbf{e}^-$ - $\textbf{e}^-$ }


\newcommand{\fromeq}[1]{\textit{equation \ref{eq:#1}}}
\newcommand{\fromeqs}[2]{\textit{equations \ref{eq:#1} and \ref{eq:#2}}}
\newcommand{\fromeqsth}[3]{\textit{equations \ref{eq:#1}, \ref{eq:#2} and \ref{eq:#3}}}
\newcommand{\fromeqsf}[4]{\textit{equations \ref{eq:#1}, \ref{eq:#2}, \ref{eq:#3} and \ref{eq:#4}}}

\newcommand{\fromfig}[1]{\textit{figure \ref{fig:#1}}}
\newcommand{\fromfigs}[2]{\textit{figures \ref{fig:#1} and \ref{fig:#2}}}
\newcommand{\fromfigf}[4]{\textit{figures \ref{fig:#1}, \ref{fig:#2}, \ref{fig:#3} and \ref{fig:#4}}}
\newcommand{\fromfigsix}[6]{\textit{figures \ref{fig:#1}, \ref{fig:#2}, \ref{fig:#3}, \ref{fig:#4}, \ref{fig:#5} and \ref{fig:#6}}}

\newcommand{\fromch}[1]{\textit{chapter \ref{ch:#1}}}

\newcommand{\fromsec}[1]{\textit{section \ref{sec:#1}}}
\newcommand{\fromsecs}[2]{\textit{sections \ref{sec:#1} and \ref{sec:#2}}}

\newcommand{\fromapp}[1]{\textit{Appendix \ref{appendix:#1}}}

\newcommand{\fromtab}[1]{\textit{Table \ref{tab:#1}}}
\newcommand{\fromtabs}[2]{\textit{Tables \ref{tab:#1} and \ref{tab:#2}}}
\newcommand{\fromtabf}[4]{\textit{Tables \ref{tab:#1}, \ref{tab:#2}, \ref{tab:#3} and \ref{tab:#4}}}



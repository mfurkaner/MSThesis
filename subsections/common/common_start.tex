\documentclass[a4paper,oneside,12pt]{report}
\usepackage{styles/fbe_tez}
\usepackage[utf8x]{inputenc} % To use Unicode (e.g. Turkish) characters
\usepackage{amsmath, amsthm} % Some extra symbols
\usepackage[bottom]{footmisc}
\usepackage{cite}
\usepackage{graphicx}
\usepackage{longtable}
\usepackage{float}
\usepackage{multirow}
\usepackage{algorithm}
\usepackage{algorithmic}
\usepackage{array}
\usepackage{amssymb,bm,cite,graphicx, fixmath, texdraw}
\usepackage{epsfig}
\usepackage{epstopdf}
\usepackage{amsmath}
\usepackage{notoccite}
\usepackage{subcaption}
\usepackage{tabu}
\usepackage{minted}
\usepackage{xcolor}

\usemintedstyle{xcode}
\definecolor{bg}{rgb}{0.95,0.95,0.95}

\interdisplaylinepenalty=2500
\hyphenation{lists} \makeatletter

\newcommand{\field}[1]{\mathbb{#1} }
\newcommand{\beq}{\begin{equation} \setlength\abovedisplayskip{5pt} 
\setlength\belowdisplayskip{5pt}}
\newcommand{\eeq}{\end{equation}}
\newcommand{\bea}{\begin{eqnarray}}
\newcommand{\eea}{\end{eqnarray}}
\newcommand{\defn}{\stackrel{\triangle}{=}}
\newcommand{\nn}{\nonumber}
\newcommand{\nnl}{\nonumber \\}
\renewcommand{\theequation}{\arabic{equation}}
\renewcommand{\labelenumi}{(\roman{enumi})}

\DeclareMathOperator*{\argmin}{arg\,min}
\DeclareMathOperator*{\argmax}{arg\,max}

\def\ifundefined{\@ifundefined}
\def\bfat{\left[ \begin{array}}
\def\emat{\end{array} \right]}
\def\bfatt{\left\{ \begin{array}}
\def\ematt{\end{array} \right.}
\def\bset{\left\{ \begin{array}}
\def\eset{\end{array} \right\}}
\def\bpar{\left( \begin{array}}
\def\epar{\end{array} \right)}

\graphicspath{{figures/}} % Graphics will be here

\newtheorem{thm}{Theorem}[chapter]
\newtheorem{prop}[thm]{Proposition}
\newtheorem{lem}[thm]{Lemma}
\newtheorem{cor}[thm]{Corollary}

\numberwithin{equation}{chapter}


% COVER PAGE
\title{DESING AND SIMULATION SOFTWARE FOR RHODOTRON TYPE ELECTRON ACCELERATORS}
\turkcebaslik{RHODOTRON TİPİ ELEKTRON HIZLANDIRICILARI İÇİN TASARIM VE SİMÜLASYON YAZILIMI}
\degree{B.S., Physics, Boğaziçi University, 2020}
\author{Muhammet Furkan Er}
\program{M.S. Physics}
\subyear{2023}

% APPROVED BY PAGE
\supervisor{Prof. Veysi Erkcan Özcan}
\cosuperi{Gökhan Ünel, Ph.D.}
%\cosuperii{Title and Name of Cosupervisor II}
\examineri{Prof. Erhan Gülmez}
\examinerii{Bora Işıldak , Ph.D.}
%\examineriv{}
%\examinerv{}
\dateofapproval{DD.MM.YYYY}

% BEGIN OF DOCS
\begin{document}
\pagenumbering{roman}
\makemstitle % M.S. thesis
\makeapprovalpage

% ACK PAGE
\begin{acknowledgements}
Acknowledgements come here...
\end{acknowledgements}

% ABS PAGE
\begin{abstract}
One page abstract will come here.  
\end{abstract}

% OZET PAGE
\begin{ozet}
Bir sayfa uzunluğunda özet gelecektir.
\end{ozet}

% CONTENTS AND LIST OF FIG AND TABLE PAGES
\tableofcontents
\listoffigures
\listoftables


\pagenumbering{arabic}



\newcommand{\vecthreeBF}[1]{\vec{\textbf{#1}}}
\newcommand{\vecthree}[1]{\vec{#1}}
\newcommand{\vecNum}[3]{(#1, #2, #3)}

\newcommand{\parDeriv}[2]{\frac{\partial #1}{\partial #2}}
\newcommand{\parDerivS}[2]{\frac{\partial^2 #1}{\partial #2^2}}
\newcommand{\derivS}[2]{\frac{d^2 #1}{d#2^2}}

\newcommand{\dotProdBF}[2]{\vecthreeBF{#1} \cdot \vecthreeBF{#2}}
\newcommand{\dotProd}[2]{\vecthree{#1} \cdot \vecthree{#2}}

\newcommand{\crossProdBF}[2]{\vecthreeBF{#1} \times \vecthreeBF{#2}}
\newcommand{\crossProd}[2]{\vecthree{#1} \times \vecthree{#2}}

\newcommand{\e}{$\textbf{e}^-$ }
\newcommand{\egun}{$\textbf{e}^-$-gun }
\newcommand{\eB}{$\textbf{e}^-$ - $\vecthreeBF{B}$ }
\newcommand{\eE}{$\textbf{e}^-$ - $\vecthreeBF{E}$ }
\newcommand{\eEM}{$\textbf{e}^-$ - \textbf{EM} }
\newcommand{\ee}{$\textbf{e}^-$ - $\textbf{e}^-$ }


\newcommand{\fromeq}[1]{\textit{equation \ref{eq:#1}}}
\newcommand{\fromeqs}[2]{\textit{equations \ref{eq:#1} and \ref{eq:#2}}}
\newcommand{\fromeqsth}[3]{\textit{equations \ref{eq:#1}, \ref{eq:#2} and \ref{eq:#3}}}
\newcommand{\fromeqsf}[4]{\textit{equations \ref{eq:#1}, \ref{eq:#2}, \ref{eq:#3} and \ref{eq:#4}}}

\newcommand{\fromfig}[1]{\textit{figure \ref{fig:#1}}}
\newcommand{\fromfigs}[2]{\textit{figures \ref{fig:#1} and \ref{fig:#2}}}
\newcommand{\fromfigf}[4]{\textit{figures \ref{fig:#1}, \ref{fig:#2}, \ref{fig:#3} and \ref{fig:#4}}}
\newcommand{\fromfigsix}[6]{\textit{figures \ref{fig:#1}, \ref{fig:#2}, \ref{fig:#3}, \ref{fig:#4}, \ref{fig:#5} and \ref{fig:#6}}}

\newcommand{\fromsec}[1]{\textit{section \ref{sec:#1}}}
\newcommand{\fromsecs}[2]{\textit{sections \ref{sec:#1} and \ref{sec:#2}}}

\newcommand{\fromapp}[1]{\textit{Appendix \ref{appendix:#1}}}

\newcommand{\fromtab}[1]{\textit{Table \ref{tab:#1}}}
\newcommand{\fromtabs}[2]{\textit{Tables \ref{tab:#1} and \ref{tab:#2}}}
\newcommand{\fromtabf}[4]{\textit{Tables \ref{tab:#1}, \ref{tab:#2}, \ref{tab:#3} and \ref{tab:#4}}}



\documentclass[a4paper,oneside,12pt]{report}
\usepackage{amsmath, amsthm} % Some extra symbols
\usepackage[bottom]{footmisc}
\usepackage{cite}
\usepackage{graphicx}
\usepackage{longtable}
\usepackage{float}
\usepackage{multirow}
\usepackage{algorithm}
\usepackage{algorithmic}
\usepackage{array}
\usepackage{amssymb,bm,cite,graphicx, fixmath, texdraw}
\usepackage{epsfig}
\usepackage{epstopdf}
\usepackage{amsmath}
\usepackage{notoccite}
\usepackage{subcaption}
\usepackage{tabu}

\interdisplaylinepenalty=2500
\hyphenation{lists} \makeatletter

\newcommand{\field}[1]{\mathbb{#1} }
\newcommand{\beq}{\begin{equation} \setlength\abovedisplayskip{5pt} 
\setlength\belowdisplayskip{5pt}}
\newcommand{\eeq}{\end{equation}}
\newcommand{\bea}{\begin{eqnarray}}
\newcommand{\eea}{\end{eqnarray}}
\newcommand{\defn}{\stackrel{\triangle}{=}}
\newcommand{\nn}{\nonumber}
\newcommand{\nnl}{\nonumber \\}
\renewcommand{\theequation}{\arabic{equation}}
\renewcommand{\labelenumi}{(\roman{enumi})}

\DeclareMathOperator*{\argmin}{arg\,min}
\DeclareMathOperator*{\argmax}{arg\,max}

\def\ifundefined{\@ifundefined}
\def\bfat{\left[ \begin{array}}
\def\emat{\end{array} \right]}
\def\bfatt{\left\{ \begin{array}}
\def\ematt{\end{array} \right.}
\def\bset{\left\{ \begin{array}}
\def\eset{\end{array} \right\}}
\def\bpar{\left( \begin{array}}
\def\epar{\end{array} \right)}

\graphicspath{{figures/}} % Graphics will be here

\newtheorem{thm}{Theorem}[chapter]
\newtheorem{prop}[thm]{Proposition}
\newtheorem{lem}[thm]{Lemma}
\newtheorem{cor}[thm]{Corollary}

\numberwithin{equation}{chapter}

% BEGIN OF DOCS
\begin{document}
\pagenumbering{roman}


% CONTENTS AND LIST OF FIG AND TABLE PAGES
\tableofcontents

\pagenumbering{arabic}

\newcommand{\vecthreeBF}[1]{\vec{\textbf{#1}}}
\newcommand{\vecthree}[1]{\vec{#1}}

\newcommand{\parDeriv}[2]{\frac{\partial #1}{\partial #2}}
\newcommand{\parDerivS}[2]{\frac{\partial^2 #1}{\partial #2^2}}
\newcommand{\derivS}[2]{\frac{d^2 #1}{d#2^2}}

\newcommand{\dotProdBF}[2]{\vecthreeBF{#1} \cdot \vecthreeBF{#2}}
\newcommand{\dotProd}[2]{\vecthree{#1} \cdot \vecthree{#2}}

\newcommand{\crossProdBF}[2]{\vecthreeBF{#1} \times \vecthreeBF{#2}}
\newcommand{\crossProd}[2]{\vecthree{#1} \times \vecthree{#2}}


\newcommand{\fromeq}[1]{\textit{equation \ref{eq:#1}}}
\newcommand{\fromeqs}[2]{\textit{equations \ref{eq:#1} and \ref{eq:#2}}}
\newcommand{\fromeqsth}[3]{\textit{equations \ref{eq:#1}, \ref{eq:#2} and \ref{eq:#3}}}

\newcommand{\fromfig}[1]{\textit{figure \ref{fig:#1}}}
\newcommand{\fromfigs}[2]{\textit{figures \ref{fig:#1} and \ref{fig:#2}}}

\newcommand{\fromsec}[1]{\textit{section \ref{sec:#1}}}
\newcommand{\fromsecs}[2]{\textit{sections \ref{sec:#1} and \ref{sec:#2}}}

\newcommand{\e}{$\textbf{e}^-$ }
\newcommand{\egun}{$\textbf{e}^-$-gun }
\newcommand{\eB}{$\textbf{e}^-$ - $\vecthreeBF{B}$ }
\newcommand{\eE}{$\textbf{e}^-$ - $\vecthreeBF{E}$ }
\newcommand{\eEM}{$\textbf{e}^-$ - \textbf{EM} }
\newcommand{\ee}{$\textbf{e}^-$ - $\textbf{e}^-$ }

%%%%%%
% START OF FUTURE_WORK

\chapter{FUTURE WORK}

At the time of publication, production efforts of a new rhodotron type electron accelerator are currently continuing in KAHVELab. 
Present bottleneck is the fabrication of the cavity.

\section{Cavity}
The cavity manufacturing process is in its concluding stages.
The sheet bars and the flanges of the upper and the lower parts have been produced and are being welded and getting ready for heat treatment.
After the heat treatment operation, 
the supporting toroidal sheets will be removed and the inside surface will be polished before cyanide copper plating. 

\section{Magnets}
Magnet production was decided to be kept on hold in the early stages of the project, 
due to the disagreement on energy gain amount between the calculations of \textit{J. POTTIER} \cite{rhodo_pottier} 
and the simulation results from \textit{CST Studio Suite} \& \textit{Rhodotron Simulation}. 
The software is planned to be used in magnet design and beam optimizations in KAHVELab after the initial tests are completed.

\section{Rhodotron Simulation}

\textit{Rhodotron Simulation}, which has been the main focus of this thesis, is ready and waiting to be tested with the cavity that is being manifactured; 
several improvements and new feature implementations are underway in the mean time. 
They include the following;
\begin{figure}[H]
    \begin{subfigure}[t]{0.5\textwidth}
        \begin{itemize}
            \item Electric and magnetic field import
            \item Field generator module to directly produce field files from specified cavity
            \item Synchrotron radiation calculations to determine radiation and energy loss
            \item \ee interactions
            \item Redesign of the GUI
            \item 3D render in GUI
        \end{itemize}
    \end{subfigure}
    \begin{subfigure}[t]{0.5\textwidth}
        \begin{itemize}
            \item Refactoring of the \eEM interaction and logging to improve performance
            \item Refactoring of the GUI Render Frame to improve render speed
            \item Refactoring the magnet class to introduce field leaks
            \item Extention of analysis tools in Analyze Frame
            \item $L_{out}$ sweep in Sweep Frame
        \end{itemize}
    \end{subfigure}
\end{figure}

\section{Conclusion}

In conclusion, the performance and the accuracy of this new tool, \textit{Rhodotron Simulation}, cannot be determined
before the cavity production is completed and the performance is compared with the predictions of the tool;
however the capabilities it presents, coupled with the analogous results obtained from other extensively utilized simulation tools, 
lead to the inference that this tool holds promise.
%%%%%%

\end{document}
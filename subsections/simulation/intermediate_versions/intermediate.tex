\documentclass{book}
\usepackage[utf8]{inputenc}
\usepackage{graphicx}
\usepackage{tikz}
\usepackage{float}
\usepackage{wrapfig,lipsum}
\usepackage{svg}
\usepackage{mathtools}
\usepackage{tabu}
\usepackage{subcaption}
\usepackage[a4paper, total={6in, 8in}]{geometry}
\usepackage{minted}

\usemintedstyle{xcode}

\definecolor{bg}{rgb}{0.95,0.95,0.95}




\begin{document}


\newcommand{\vecthreeBF}[1]{\vec{\textbf{#1}}}
\newcommand{\vecthree}[1]{\vec{#1}}

\newcommand{\parDeriv}[2]{\frac{\partial #1}{\partial #2}}
\newcommand{\parDerivS}[2]{\frac{\partial^2 #1}{\partial #2^2}}
\newcommand{\derivS}[2]{\frac{d^2 #1}{d#2^2}}

\newcommand{\dotProdBF}[2]{\vecthreeBF{#1} \cdot \vecthreeBF{#2}}
\newcommand{\dotProd}[2]{\vecthree{#1} \cdot \vecthree{#2}}

\newcommand{\crossProdBF}[2]{\vecthreeBF{#1} \times \vecthreeBF{#2}}
\newcommand{\crossProd}[2]{\vecthree{#1} \times \vecthree{#2}}


\newcommand{\fromeq}[1]{\textit{equation \ref{eq:#1}}}
\newcommand{\fromeqs}[2]{\textit{equations \ref{eq:#1} and \ref{eq:#2}}}
\newcommand{\fromeqsth}[3]{\textit{equations \ref{eq:#1}, \ref{eq:#2} and \ref{eq:#3}}}

\newcommand{\fromfig}[1]{\textit{figure \ref{fig:#1}}}
\newcommand{\fromfigs}[2]{\textit{figures \ref{fig:#1} and \ref{fig:#2}}}

\newcommand{\fromsec}[1]{\textit{section \ref{sec:#1}}}
\newcommand{\fromsecs}[2]{\textit{sections \ref{sec:#1} and \ref{sec:#2}}}

\newcommand{\fromapp}[1]{\textit{Appendix \ref{appendix:#1}}}

%----../../..++++.


%%%%%%
% Start of intermediate.tex
\section{Intermediate Versions}

\subsection{$L_{out}$ Optimization For Single $e^-$}
Initial step of improving $POC$ towards $Rhodotron Simulation$ was to implement \textit{$L_{out}$ optimizations} to help optimazing magnet designs, as discussed in \fromsec{parameter_sweep}.
First approach was to hang the $e^-$ outside of the cavity for $t_{out} = L_{out}/v$, then inject it back to the cavity with reversed $\vecthreeBF{v}$. Then sweep the $t_{out}$ parameter to find the optimal value.
This simple implementation can be found in \fromfig{lout_opt_single_e} of \fromapp{intermediate_codes}.

Although the results from this optimization sweep were promising after they were simulated with $CST$, simulating one particle would not be sufficiently useful for designing a magnet.

\subsection{From Single $e^-$ to Bunches}
After successfully accelerating single $e^-$, next step was to implement bunches to approximate a real $e^-$ gun.

%%%%%%

\appendix
\chapter{Intermediate Codes} \label{appendix:intermediate_codes}
\begin{figure}[H]
    \centering
    \begin{subfigure}{\textwidth}
        \centering
        \begin{minted}[linenos=true, autogobble, frame=lines, framesep=2mm]{c++}
            for(double i = 2; i < 9; i += dT_out){
                t_dum += i;
                double enow = gecis(r_pos, vel, Et, t_dum);
                if( enow > maxE ){
                  maxE = enow;
                  t_opt = i;
                }
                t_dum = t;
            }
        \end{minted}
    \end{subfigure} 
    \\
    \begin{subfigure}{\textwidth}
        \centering
        \begin{minted}[linenos=true, autogobble, frame=lines, framesep=2mm]{c++}
            double gecis(double r_pos, double vel, double Et, double &t){
                for(; r_pos >= -R2 && r_pos <= R2 ; t+=dT){
                    vel = c*sqrt(Et*Et-E0*E0)/Et;
                    double RelBeta  = vel/c;
                    double RelGamma = 1.0 / sqrt(1.0-RelBeta*RelBeta);
                
                    double ef=Eradial(r_pos*1000,t,RFphase*deg_to_rad);
                
                    double acc=ef*1E6*eQMratio/(RelGamma*RelGamma*RelGamma); 
                
                    r_pos = r_pos + vel * dT*ns + 1/2*acc*(dT*ns)*(dT*ns);
                    vel=vel+acc*dT*ns;
                    RelBeta  = vel/c;
                    RelGamma = 1.0 / sqrt(1.0-RelBeta*RelBeta);
                    Et=RelGamma*E0; 
                }
                return Et;
            }
        \end{minted}
    \end{subfigure}
    \caption{$L_{out}$ Optimization For Single $e^-$}
    \label{fig:lout_opt_single_e}
\end{figure}



\end{document}
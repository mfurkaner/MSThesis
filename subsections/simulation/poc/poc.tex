\documentclass{article}
\usepackage[utf8]{inputenc}
\usepackage{graphicx}
\usepackage{tikz}
\usepackage{float}
\usepackage{wrapfig,lipsum}
\usepackage{svg}
\usepackage{mathtools}
\usepackage{tabu}
\usepackage{subcaption}
\usepackage[a4paper, total={6in, 8in}]{geometry}
\usepackage{minted}


\begin{document}


\newcommand{\vecthreeBF}[1]{\vec{\textbf{#1}}}
\newcommand{\vecthree}[1]{\vec{#1}}

\newcommand{\parDeriv}[2]{\frac{\partial #1}{\partial #2}}
\newcommand{\parDerivS}[2]{\frac{\partial^2 #1}{\partial #2^2}}
\newcommand{\derivS}[2]{\frac{d^2 #1}{d#2^2}}

\newcommand{\dotProdBF}[2]{\vecthreeBF{#1} \cdot \vecthreeBF{#2}}
\newcommand{\dotProd}[2]{\vecthree{#1} \cdot \vecthree{#2}}

\newcommand{\crossProdBF}[2]{\vecthreeBF{#1} \times \vecthreeBF{#2}}
\newcommand{\crossProd}[2]{\vecthree{#1} \times \vecthree{#2}}


\newcommand{\fromeq}[1]{\textit{equation \ref{eq:#1}}}
\newcommand{\fromeqs}[2]{\textit{equations \ref{eq:#1} and \ref{eq:#2}}}
\newcommand{\fromeqsth}[3]{\textit{equations \ref{eq:#1}, \ref{eq:#2} and \ref{eq:#3}}}

\newcommand{\fromfig}[1]{\textit{figure \ref{fig:#1}}}
\newcommand{\fromfigs}[2]{\textit{figures \ref{fig:#1} and \ref{fig:#2}}}

\newcommand{\fromsec}[1]{\textit{section \ref{sec:#1}}}
\newcommand{\fromsecs}[2]{\textit{sections \ref{sec:#1} and \ref{sec:#2}}}

%----../../..++++.


%%%%%%
% Start of poc.tex
\section{Proof of Concept}
The POC worked with interacting an electron with the EM field provided, at discrete time steps $dt$. The core sequence of this software is provided below.
Where $\vecthreeBF{r}$ is the position, $\vecthreeBF{v}$ is the velocity of the $e^-$;
\begin{itemize}
    \item Get electron energy $E_{in}$
    \item Get the RF field definitions
    \item Get the magnet design parameters if there is any
    \item Do following until the simulation time has ellapsed
    \begin{enumerate}
        \item Calculate $\vecthreeBF{E}_{||}$($\vecthreeBF{r}$)
        \item Calculate $\vecthreeBF{a}_{E ||}$ using \fromeq{para_and_perp_acc_of_lorentz_force}
        \item Calculate $\vecthreeBF{r}$($t+dt$) using \fromeq{leapfrog_sync_x}
        \item Calculate $\vecthreeBF{v}$($t+dt$) using \fromeq{leapfrog_sync_v}
        \item Calculate new $E_t$
    \end{enumerate}
\end{itemize}
The implemenation of this logic can be observed in \fromfig{POC_core_logic}. 
\begin{figure}[H]
    \begin{minted}[linenos=true, autogobble, frame=lines, framesep=2mm, fontsize=\footnotesize]{c++}
        for (double t=0; t<SimuTime; t+=dT){
            double RelBeta  = v/c;
            double RelGamma = 1.0 / sqrt(1.0-RelBeta*RelBeta);
        
            double ef=Eradial(r_pos*1000,t,0); // convert position to mm
            double acc=ef*1E6*eQMratio/(RelGamma*RelGamma*RelGamma); 
        
            r_pos = r_pos + v * dT*ns + 1/2*acc*(dT*ns)*(dT*ns);
            v = v + acc*dT*ns;

            RelBeta  = v/c;
            RelGamma = 1.0 / sqrt(1.0-RelBeta*RelBeta);
            Et=RelGamma*E0; 
        }
    \end{minted}
    \caption{Core logic loop of the POC}
    \label{fig:POC_core_logic}
\end{figure}
After simulating the \textit{synchronous electron} mentioned in \fromsec{cavity_of_a_rhodotron} with $\phi_{lag}=15^\circ$ for one pass, the results from \textit{POC}, \textit{CST} and \fromeq{W_gain_each_pass_pottier} are compared.
% 40 kW total power
% Emax = 0.96 in old, 2.08 in new version
% R1 = 0.1882
% R2 = 0.753
% f = 107.5
\begin{figure}[H]
    \begin{eqnarray}
        E_{Pottier} &=& 0.565 MeV \nonumber\\
        E_{CST} &=& 0.872 MeV  \label{eq:poc_E_results}\\
        E_{POC} &=& 0.873 MeV \nonumber
    \end{eqnarray}
    \caption*{$P=40$kW, $R_1=0.188$m, $R_2=0.753$m, $f=107.5$MHz}
\end{figure}
As can be observed in \fromeq{poc_E_results}, \textit{POC software} and \textit{CST} produced very similar results. However, we cannot make the same claim regarding \fromeq{W_gain_each_pass_pottier}.
This inconsistency between $E_{Pottier}$ and $E_{CST}$ was noticed in the earlier simulations as well, which supported the idea of using another simulation software to reduce reliance on \textit{CST}.
Since using a single software would have made the process error prone.

After the results obtained from \textit{POC} were found to be promising, a decision was made to develop a more robust simulation software. This software was named \textit{Rhodotron Simulation} for the current stage and started development in October, 2021.
The development and improvement efforts are still ongoing. 

During the development of \textit{Rhodotron Simulation} software, several intermediate builds have been implemented and tested. The following sections in this chapter investigates their implementations further.


%%%%%%

-lout opt
-phlag opt
-going 3D
-implement actual magnets instead of relying on lout
-


\end{document}
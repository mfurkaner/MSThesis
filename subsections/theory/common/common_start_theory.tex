\documentclass[a4paper,oneside,12pt]{report}
\usepackage{styles/fbe_tez}
\usepackage[utf8x]{inputenc} % To use Unicode (e.g. Turkish) characters
\usepackage{amsmath, amsthm} % Some extra symbols
\usepackage[bottom]{footmisc}
\usepackage{cite}
\usepackage{graphicx}
\usepackage{longtable}
\usepackage{float}
\usepackage{multirow}
\usepackage{algorithm}
\usepackage{algorithmic}
\usepackage{array}
\usepackage{amssymb,bm,cite,graphicx, fixmath, texdraw}
\usepackage{epsfig}
\usepackage{epstopdf}
\usepackage{amsmath}
\usepackage{notoccite}
\usepackage{subcaption}
\usepackage{tabu}

\interdisplaylinepenalty=2500
\hyphenation{lists} \makeatletter

\newcommand{\field}[1]{\mathbb{#1} }
\newcommand{\beq}{\begin{equation} \setlength\abovedisplayskip{5pt} 
\setlength\belowdisplayskip{5pt}}
\newcommand{\eeq}{\end{equation}}
\newcommand{\bea}{\begin{eqnarray}}
\newcommand{\eea}{\end{eqnarray}}
\newcommand{\defn}{\stackrel{\triangle}{=}}
\newcommand{\nn}{\nonumber}
\newcommand{\nnl}{\nonumber \\}
\renewcommand{\theequation}{\arabic{equation}}
\renewcommand{\labelenumi}{(\roman{enumi})}

\DeclareMathOperator*{\argmin}{arg\,min}
\DeclareMathOperator*{\argmax}{arg\,max}

\def\ifundefined{\@ifundefined}
\def\bfat{\left[ \begin{array}}
\def\emat{\end{array} \right]}
\def\bfatt{\left\{ \begin{array}}
\def\ematt{\end{array} \right.}
\def\bset{\left\{ \begin{array}}
\def\eset{\end{array} \right\}}
\def\bpar{\left( \begin{array}}
\def\epar{\end{array} \right)}

\graphicspath{{figures/}} % Graphics will be here

\newtheorem{thm}{Theorem}[chapter]
\newtheorem{prop}[thm]{Proposition}
\newtheorem{lem}[thm]{Lemma}
\newtheorem{cor}[thm]{Corollary}

\numberwithin{equation}{chapter}

% BEGIN OF DOCS
\begin{document}
\pagenumbering{roman}


% CONTENTS AND LIST OF FIG AND TABLE PAGES
\tableofcontents

\pagenumbering{arabic}
\newcommand{\fromeq}[1]{\textit{equation \ref{eq:#1}}}
\newcommand{\fromeqs}[2]{\textit{equations \ref{eq:#1} and \ref{eq:#2}}}

\newcommand{\vecthreeBF}[1]{\vec{\textbf{#1}}}
\newcommand{\vecthree}[1]{\vec{#1}}

\newcommand{\parDeriv}[2]{\frac{\partial #1}{\partial #2}}
\newcommand{\parDerivS}[2]{\frac{\partial^2 #1}{\partial #2^2}}
\newcommand{\derivS}[2]{\frac{d^2 #1}{d#2^2}}

\newcommand{\dotProdBF}[2]{\vecthreeBF{#1} \cdot \vecthreeBF{#2}}
\newcommand{\dotProd}[2]{\vecthree{#1} \cdot \vecthree{#2}}

\newcommand{\crossProdBF}[2]{\vecthreeBF{#1} \times \vecthreeBF{#2}}
\newcommand{\crossProd}[2]{\vecthree{#1} \times \vecthree{#2}}

\newcommand{\fromfig}[1]{\textit{figure \ref{fig:#1}}}

%#####
% START OF INTRODUCTION AND THEORY

\chapter{INTRODUCTION AND THEORY}

Particle accelerators are essential tools for scientific research. They enable scientists to probe the fundamental constituents of matter, study particle interactions, and explore the laws of physics. 
Accelerators have been instrumental in discovering and characterizing fundamental particles, such as quarks, leptons, and more recently the Higgs boson. 
They also facilitate the production of high-intensity particle beams for applications in industrial processes. A conservative estimate puts the number of industrial high-current electron
accelerators to be in the thousands, worldwide.

MeV-level electron beams, both as is and as X-Ray sources, are especially useful for industrial and medical applications. 
For instance,  high-energy beams (ranging from 5 to 10 MeV) find application in medical accelerators and in services such as sterilizing medical devices and irradiating food. 
Meanwhile, medium-energy devices (ranging from 400 keV to 5 MeV) contribute to the improvement of wires, cables, and tires, and also play a role in treating factory flue gases and wastewater.
Although most of the industrial electron accelerators operating at MeV-level are linear accelerators at the moment, due to their compact nature, circular accelerators have been gaining prominence. 

Rose-pattle accelerator is a type of circular accelerator suitable for producing electron beams at MeV-level energy. 
It was developed in 1989 and was named \textit{Rhodotron} \cite{rhodo_pottier}. This name was trademarked by \textit{IBA Group} in the following years. 



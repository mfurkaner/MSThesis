\documentclass{article}

\usepackage[utf8]{inputenc}
\usepackage{graphicx}
\usepackage{tikz}
\usepackage{float}
\usepackage{wrapfig,lipsum}
\usepackage{svg}
\usepackage{mathtools}
\usepackage{tabu}
\usepackage[a4paper, total={6in, 8in}]{geometry}

\begin{document}


\newcommand{\vecthreeBF}[1]{\vec{\textbf{#1}}}
\newcommand{\vecthree}[1]{\vec{#1}}

\newcommand{\parDeriv}[2]{\frac{\partial #1}{\partial #2}}
\newcommand{\parDerivS}[2]{\frac{\partial^2 #1}{\partial #2^2}}
\newcommand{\derivS}[2]{\frac{d^2 #1}{d#2^2}}

\newcommand{\dotProdBF}[2]{\vecthreeBF{#1} \cdot \vecthreeBF{#2}}
\newcommand{\dotProd}[2]{\vecthree{#1} \cdot \vecthree{#2}}

\newcommand{\crossProdBF}[2]{\vecthreeBF{#1} \times \vecthreeBF{#2}}
\newcommand{\crossProd}[2]{\vecthree{#1} \times \vecthree{#2}}


\newcommand{\fromeq}[1]{\textit{equation \ref{eq:#1}}}
\newcommand{\fromeqs}[2]{\textit{equations \ref{eq:#1} and \ref{eq:#2}}}


\section{Theory}

\subsection{Numerical Integration Methods}

\subsubsection{Leap Frog}
Leapfrog is a method that is used to numarically intergenerate that are in the form of

\begin{equation*}
    \ddot{x} = \derivS{x}{t} = f(x)
\end{equation*}

It is also known as the Störmer-Verlet method, commonly used to numerically calculate the trejectory of particles. 
The name comes from the fact that calculation of updated $\textbf{x}$ and $\textbf{v}$ are done in some order and calculated for different time slices. 
The energy is approximately conserved during the calculation. 
It is stable in oscillatory motion as long as the time-step $\Delta t$ is constant, and satisfies $\Delta t\leq 2/\omega$ \cite{leapfrog_osc}.
The idea is straight forward; in the time interval $\Delta t$, 

\begin{eqnarray}
    a(t_0) &=& f(x_0) \nonumber \\
    x(t_0 + \Delta t) &=& x(t_0) + v(t_0)\Delta t + a(t_0)\frac{\Delta t^2}{2} \label{eq:leapfrog_sync_x}\\
    v(t_0 + \Delta t) &=& v(t_0) + \{ a(t_0) + a(t_0 + \Delta t)\}\frac{\Delta t}{2}  \label{eq:leapfrog_sync_v}
\end{eqnarray}

For more stability, this version can be rearranged to what is called 'kick-drift-kick' form,

\begin{eqnarray} \label{eq:leapfrog}
    v(t_0 + \Delta t/2) &=& v(t_0) +  a(t_0)\frac{\Delta t}{2} \nonumber \\
    x(t_0 + \Delta t) &=& x(t_0) + v(t_0 + \Delta t/2)\Delta t \\
    v(t_0 + \Delta t) &=& v(t_0 + \Delta t/2) + a(t_0 + \Delta t)\frac{\Delta t}{2}  \nonumber
\end{eqnarray}

This version provides more time resolution to our calculation; however, it increases the number of calculations needed by about $50\%$.

\subsubsection{Runge Kutta}

Runge Kutta is another numerical integration method that is 



\begin{thebibliography}{9}
\bibitem{leapfrog_osc}
C. K. Birdsall and A. B. Langdon, \emph{Plasma Physics via Computer Simulations}, McGraw-Hill Book Company, 1985, p. 56.
\end{thebibliography}

\end{document}
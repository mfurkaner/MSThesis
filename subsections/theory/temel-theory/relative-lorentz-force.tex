\documentclass{article}
\usepackage[utf8]{inputenc}
\usepackage{graphicx}
\usepackage{tikz}
\usepackage{float}
\usepackage{wrapfig,lipsum}
\usepackage{svg}
\usepackage{mathtools}
\usepackage{tabu}
\usepackage[a4paper, total={6in, 8in}]{geometry}

\begin{document}


\newcommand{\vecthreeBF}[1]{\vec{\textbf{#1}}}
\newcommand{\vecthree}[1]{\vec{#1}}

\newcommand{\parDeriv}[2]{\frac{\partial #1}{\partial #2}}

\newcommand{\dotProdBF}[2]{\vecthreeBF{#1} \cdot \vecthreeBF{#2}}
\newcommand{\dotProd}[2]{\vecthree{#1} \cdot \vecthree{#2}}

\newcommand{\crossProdBF}[2]{\vecthreeBF{#1} \times \vecthreeBF{#2}}
\newcommand{\crossProd}[2]{\vecthree{#1} \times \vecthree{#2}}


\newcommand{\fromeq}[1]{\textit{equation \ref{eq:#1}}}
\newcommand{\fromeqs}[2]{\textit{equations \ref{eq:#1} and \ref{eq:#2}}}

%%%%%%

\section{Accelerating Charged Particles}

\subsection{Relation between momentum and acceleration}
In classical mechanics, Newton's second law defines force and states 
\begin{equation} \label{eq:newton_second_law}
    \vecthreeBF{F} = \frac{\partial \vecthreeBF{p}}{\partial t} = m \frac{\partial \vecthreeBF{v}}{\partial t} = m \vecthreeBF{a}
\end{equation}
In special relativity however, relativistic momentum is defined as $\vecthree{p} = \gamma m_0 \vecthree{v} $ where $\gamma$ is the Lorentz Factor;
\begin{equation*}
    \gamma = \frac{1}{\sqrt{1-v^2 / c^2}} = \frac{1}{\sqrt{1-\beta^2}}
\end{equation*}
Considering these two statements, we can find the relation between momentum and acceleration as follows
\begin{eqnarray*}
    \parDeriv{\vecthree{p}}{t} &=& m_0 \parDeriv{(\gamma \vecthree{v})}{t} = m_0 \{  \parDeriv{\gamma}{t}\vecthree{v} + \gamma \parDeriv{\vecthree{v}}{t}  \} \\
    \parDeriv{\gamma}{t} &=& \gamma^3 \dotProd{\beta}{\parDeriv{\beta}{t}} = \frac{\gamma^3}{c} \dotProd{\beta}{a} \\
    \parDeriv{\vecthree{p}}{t}  &=& m_0 \{   \frac{\gamma^3}{c} (\dotProd{\beta}{a} )\vecthree{v} + \gamma \parDeriv{\vecthree{v}}{t}  \} 
\end{eqnarray*}
\begin{equation} \label{eq:Frel}
    \vecthreeBF{F}  = \gamma m_0 \{ \vecthreeBF{a} + \gamma^2(\dotProd{\beta}{\textbf{a}})\vecthree{\beta} \}
\end{equation}
It is clear that acceleration is not necessarily parallel to the force anymore. To start seperating the parallel and perpendicular components relative to $\vecthree{\beta}$, we can find $\vecthreeBF{a}_{||}$ and  $\vecthreeBF{F}_{||}$;
\begin{equation} \label{eq:a_F_parallels}
    \begin{aligned}
        \vecthreeBF{a}_{||} = \frac{(\dotProd{a}{\beta})}{\beta^2}\beta 
    \end{aligned}
    \qquad \qquad
    \begin{aligned}
        \vecthreeBF{F}_{||} = \frac{(\dotProd{F}{\beta})}{\beta^2}\beta 
    \end{aligned}
\end{equation}
\begin{eqnarray*}
    \dotProd{\textbf{F}}{\beta} &=& \gamma m_0 \{ \dotProd{\textbf{a}}{\beta} + \gamma^2(\dotProd{\beta}{\textbf{a}})\beta^2 \} \\
                                &=& \gamma m_0 (\dotProd{\textbf{a}}{\beta}) \{ \gamma^2\beta^2  + 1\} 
\end{eqnarray*}
Using $\gamma^2 \beta^2 + 1 = \gamma^2 $ we have,
\begin{equation*}
    \dotProd{\textbf{F}}{\beta} = m_0 \gamma^3 (\dotProd{\textbf{a}}{\beta} )
\end{equation*}
Inserting this
\begin{eqnarray}
    \vecthreeBF{F}_{||} &=& \frac{(\dotProd{\textbf{F}}{\beta})}{\beta^2} \vec{\beta} \nonumber \\
                        &=& m_0 \gamma^3 \frac{(\dotProd{\textbf{a}}{\beta} )}{\beta^2} \vec{\beta} \nonumber \\
                        &=& m_0 \gamma^3 \vecthreeBF{a}_{||} \label{eq:Frel_parallel}
\end{eqnarray}
Therefore from \fromeqs{Frel}{Frel_parallel}
\begin{eqnarray*}
    \vecthreeBF{F}  &=& m_0 \gamma^3 \vecthreeBF{a}_{||} \beta^2 + m_0 \gamma \vecthreeBF{a} \\
                    &=& m_0 \gamma^3 \vecthreeBF{a}_{||} \beta^2 + m_0 \gamma \{ \vecthreeBF{a}_{||} +  \vecthreeBF{a}_{\perp} \} \\
                    &=& m_0 \vecthreeBF{a}_{||} \gamma \{ \gamma^2\beta^2 + 1 \} + m_0 \gamma \vecthreeBF{a}_{\perp} \\
                    &=& m_0 \vecthreeBF{a}_{||} \gamma^3 + m_0 \gamma \vecthreeBF{a}_{\perp} \\
                    &=& \vecthreeBF{F}_{||} + m_0 \gamma \vecthreeBF{a}_{\perp}
\end{eqnarray*}
We finally have two seperate equations which are in similar form with \fromeq{newton_second_law}.
\begin{equation} \label{eq:Frel_para_and_perp}
    \begin{aligned}
        \vecthreeBF{F}_{||} = \gamma^3  m_0 \vecthreeBF{a}_{||}
    \end{aligned}
    \qquad \qquad
    \begin{aligned}
        \vecthreeBF{F}_{\perp} = \gamma  m_0\vecthreeBF{a}_{\perp}
    \end{aligned}
\end{equation}


\subsection{Lorentz Force} \label{sec:lorentz-force}
Force acting on a charged particle moving in electromagnetic fields is called Lorentz Force and is given by the formula
\newline
\begin{equation} \label{eq:lf}
    \frac{\partial \vecthree{p}}{\partial t} = \vecthree{F}_L=q(\vecthree{E}+ \vecthree{v} \times \vecthree{B})
\end{equation}
\newline
where the $q$ is the charge and $\vecthree{v}$ is the velocity of the particle. 

\subsection{Relativistic Lorentz Force}
Similar to non-relativistic version, relativistic Lorentz Force is given by the following 4-vector equality
\begin{equation}
    \frac{\partial p^{\mu}}{\partial \tau} = q F^{\mu \nu} u_{\nu}
\end{equation}
Where $ \partial \tau = \partial t / \gamma $,
\begin{equation}
    \begin{aligned}
        p^{\mu} = 
        \begin{bmatrix}
            W/c \\
            p_x         \\
            p_y \\
            p_z
        \end{bmatrix}    
    \end{aligned}
    \qquad\qquad
    \begin{aligned}
        F^{\mu\nu} = 
        \begin{bmatrix}
                0       & -E_x/c   & -E_y/c    & -E_z/c \\
                E_x/c   &   0      & -B_z      & B_y     \\
                E_y/c   &   B_z    &  0        & -B_x     \\
                E_z/c   &   -B_y   & B_x       & 0   
        \end{bmatrix} 
    \end{aligned}
    \qquad\qquad
    \begin{aligned}
        u_{\nu} = \gamma
        \begin{bmatrix}
                 c \\
                -v_x \\
                -v_y \\
                -v_z \\
        \end{bmatrix} 
    \end{aligned}
\end{equation}
Where $W$ is the energy of the particle and $\gamma$ is the lorentz factor mentioned in the previous section.
\newline

For $ \mu = 0 $, we have the time component of the equation;
\begin{eqnarray}
    \frac{\gamma}{c}\frac{\partial W}{\partial t} &=& \frac{q \gamma \vecthree{E}\cdot \vecthree{v}}{c} = \frac{q \gamma}{c}  \frac{\vecthree{E}  \cdot \partial \vecthree{r}}{\partial t} \\
    \frac{\partial W}{\partial t} &=& q \frac{\vecthree{E}\cdot \partial \vecthree{r}} {\partial t}
\end{eqnarray}
This is the definition of work done by an electric field. For $ \mu = 1,2,3 $, we have the spacial components;
\begin{equation*}
    \frac{\partial \vecthree{p}}{\partial \tau} = \gamma \frac{\partial \vecthree{p}}{\partial t} = q \gamma (\vecthree{E} + \vecthree{v} \times \vecthree{B})
\end{equation*}
Which simplifies to non-relativistic Lorentz Force in \fromeq{lf}.


\subsection{Acceleration caused by lorentz force}
Due to the nature of the cross product, lorentz force caused by a magnetic field is always perpendicular to the velocity of the particle. 
Therefore the acceleration of the magnetic field is straightforward
\begin{equation*}
    \vecthreeBF{F}_B = \vecthreeBF{F}_{\perp} = \gamma  m_0\vecthreeBF{a}_{\perp} = \gamma  m_0\vecthreeBF{a}_{B}
\end{equation*}
The same thing cannot be said about electric field however. 
It can create force in any direction with respect to velocity. Therefore, we have the following equality;
\begin{equation} \label{eq:para_and_perp_acc_of_lorentz_force}
    \begin{aligned}
        \vecthreeBF{a}_{||} = \frac{q }{\gamma^3 m_0} \vecthreeBF{E}_{||}
    \end{aligned}
    \qquad \qquad
    \begin{aligned}
        \vecthreeBF{a}_{\perp} = \frac{q}{\gamma m_0} \{ \vecthreeBF{E}_{\perp} + \crossProdBF{v}{B}   \}
    \end{aligned}    
\end{equation}
Acceleration due to electric field can be simplified as;
\begin{eqnarray*}
    \vecthreeBF{a}_{\{B=0\}} = \vecthreeBF{a}_{E}   &=&  \vecthreeBF{a}_{||} + \vecthreeBF{a}_{\perp \{B=0\}} \\
                                &=& \frac{q}{m_0 \gamma} \{ \frac{\vecthreeBF{E}_{||}}{\gamma^2} + \vecthreeBF{E}_{\perp} \} \\
                                &=& \frac{q}{m_0 \gamma} \{ \{1 - \beta^2\}\vecthreeBF{E}_{||} + \vecthreeBF{E}_{\perp} \} \\
                                &=& \frac{q}{m_0 \gamma} \{ \vecthreeBF{E}_{||} + \vecthreeBF{E}_{\perp}-\beta^2\vecthreeBF{E}_{||} \} \\
                                &=& \frac{q}{m_0 \gamma} \{ \vecthreeBF{E} - \beta^2\vecthreeBF{E}_{||} \} \\
\end{eqnarray*}
Using the fact that $\vecthreeBF{E}_{||} = \vec{\beta}(\dotProd{\textbf{E}}{\beta})/\beta^2 $, we finally have
\begin{equation} \label{eq:acc_of_E_and_B}
    \begin{aligned}
        \vecthreeBF{a}_{E} = \frac{q}{ \gamma m_0} \{ \vecthreeBF{E} - \vecthreeBF{v}\frac{(\dotProdBF{E}{v})}{c^2} \}
    \end{aligned}
    \qquad \qquad
    \begin{aligned}
        \vecthreeBF{a}_{B} = \frac{q}{\gamma m_0} ( \crossProdBF{v}{B} )
    \end{aligned}
\end{equation}
%%%%%%

\end{document}
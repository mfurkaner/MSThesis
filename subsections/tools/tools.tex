\documentclass[a4paper,oneside,12pt]{report}
\usepackage{amsmath, amsthm} % Some extra symbols
\usepackage[bottom]{footmisc}
\usepackage{cite}
\usepackage{graphicx}
\usepackage{longtable}
\usepackage{float}
\usepackage{multirow}
\usepackage{algorithm}
\usepackage{algorithmic}
\usepackage{array}
\usepackage{amssymb,bm,cite,graphicx, fixmath, texdraw}
\usepackage{epsfig}
\usepackage{epstopdf}
\usepackage{amsmath}
\usepackage{notoccite}
\usepackage{subcaption}
\usepackage{tabu}

\interdisplaylinepenalty=2500
\hyphenation{lists} \makeatletter

\newcommand{\field}[1]{\mathbb{#1} }
\newcommand{\beq}{\begin{equation} \setlength\abovedisplayskip{5pt} 
\setlength\belowdisplayskip{5pt}}
\newcommand{\eeq}{\end{equation}}
\newcommand{\bea}{\begin{eqnarray}}
\newcommand{\eea}{\end{eqnarray}}
\newcommand{\defn}{\stackrel{\triangle}{=}}
\newcommand{\nn}{\nonumber}
\newcommand{\nnl}{\nonumber \\}
\renewcommand{\theequation}{\arabic{equation}}
\renewcommand{\labelenumi}{(\roman{enumi})}

\DeclareMathOperator*{\argmin}{arg\,min}
\DeclareMathOperator*{\argmax}{arg\,max}

\def\ifundefined{\@ifundefined}
\def\bfat{\left[ \begin{array}}
\def\emat{\end{array} \right]}
\def\bfatt{\left\{ \begin{array}}
\def\ematt{\end{array} \right.}
\def\bset{\left\{ \begin{array}}
\def\eset{\end{array} \right\}}
\def\bpar{\left( \begin{array}}
\def\epar{\end{array} \right)}

\graphicspath{{figures/}} % Graphics will be here

\newtheorem{thm}{Theorem}[chapter]
\newtheorem{prop}[thm]{Proposition}
\newtheorem{lem}[thm]{Lemma}
\newtheorem{cor}[thm]{Corollary}

\numberwithin{equation}{chapter}

% BEGIN OF DOCS
\begin{document}
\pagenumbering{roman}


% CONTENTS AND LIST OF FIG AND TABLE PAGES
\tableofcontents

\pagenumbering{arabic}

\newcommand{\vecthreeBF}[1]{\vec{\textbf{#1}}}
\newcommand{\vecthree}[1]{\vec{#1}}

\newcommand{\parDeriv}[2]{\frac{\partial #1}{\partial #2}}
\newcommand{\parDerivS}[2]{\frac{\partial^2 #1}{\partial #2^2}}
\newcommand{\derivS}[2]{\frac{d^2 #1}{d#2^2}}

\newcommand{\dotProdBF}[2]{\vecthreeBF{#1} \cdot \vecthreeBF{#2}}
\newcommand{\dotProd}[2]{\vecthree{#1} \cdot \vecthree{#2}}

\newcommand{\crossProdBF}[2]{\vecthreeBF{#1} \times \vecthreeBF{#2}}
\newcommand{\crossProd}[2]{\vecthree{#1} \times \vecthree{#2}}


\newcommand{\fromeq}[1]{\textit{equation \ref{eq:#1}}}
\newcommand{\fromeqs}[2]{\textit{equations \ref{eq:#1} and \ref{eq:#2}}}
\newcommand{\fromeqsth}[3]{\textit{equations \ref{eq:#1}, \ref{eq:#2} and \ref{eq:#3}}}

\newcommand{\fromfig}[1]{\textit{figure \ref{fig:#1}}}
\newcommand{\fromfigs}[2]{\textit{figures \ref{fig:#1} and \ref{fig:#2}}}

\newcommand{\fromsec}[1]{\textit{section \ref{sec:#1}}}
\newcommand{\fromsecs}[2]{\textit{sections \ref{sec:#1} and \ref{sec:#2}}}

\newcommand{\e}{$\textbf{e}^-$ }
\newcommand{\egun}{$\textbf{e}^-$-gun }
\newcommand{\eB}{$\textbf{e}^-$ - $\vecthreeBF{B}$ }
\newcommand{\eE}{$\textbf{e}^-$ - $\vecthreeBF{E}$ }
\newcommand{\eEM}{$\textbf{e}^-$ - \textbf{EM} }
\newcommand{\ee}{$\textbf{e}^-$ - $\textbf{e}^-$ }

%%%%%%
% START OF tools.tex

\chapter{TOOLS}

\section{Poisson Superfish}
Poisson Superfish is a software package developed by the \textit{Los Alamos National Laboratory} 
for the design and analysis of electromagnetic fields, particularly those related to particle accelerators and high-energy physics experiments.
It is commonly used in the field of accelerator physics to simulate and optimize the behavior of charged particle beams as they pass through various electromagnetic structures, such as cavities and magnets.


\section{CST Studio Suite}
CST Studio Suite is a powerful software package developed by \textit{Simulia} for electromagnetic simulation and analysis. It's widely used in various industries, including electronics, telecommunications, automotive, aerospace, and more, to design and optimize electromagnetic devices and systems. 
The software provides a comprehensive set of tools for simulating and analyzing the behavior of electromagnetic fields and their interactions with different materials, structures and charged particles.

\section{ROOT}
ROOT is a widely used open-source software framework developed by CERN (European Organization for Nuclear Research) for data analysis,
visualization, and storage in the field of high-energy physics, particularly in experiments conducted at particle accelerators 
like the Large Hadron Collider (LHC). It has become an essential tool in particle physics research, 
and it's used by physicists around the world to analyze and interpret data.

\section{gnuplot}
GNUPLOT is a popular open-source software used for creating and visualizing data plots and graphs. 
It is widely utilized in various fields, including scientific research, data analysis, engineering, and more, 
to represent and analyze data in a graphical format. 

%%%%%%

\end{document}
\documentclass[a4paper,oneside,12pt]{report}
\usepackage{styles/fbe_tez}
\usepackage[utf8x]{inputenc} % To use Unicode (e.g. Turkish) characters
\usepackage{amsmath, amsthm} % Some extra symbols
\usepackage[bottom]{footmisc}
\usepackage{cite}
\usepackage{graphicx}
\usepackage{longtable}
\usepackage{float}
\usepackage{multirow}
\usepackage{subfigure}
\usepackage{algorithm}
\usepackage{algorithmic}
\usepackage{array}
\usepackage{amssymb,bm,cite,graphicx, fixmath, texdraw}
\usepackage{epsfig}
\usepackage{epstopdf}
\usepackage{subfigure} % 
\usepackage{amsmath}
\usepackage{notoccite}

\interdisplaylinepenalty=2500
\hyphenation{lists} \makeatletter

\newcommand{\field}[1]{\mathbb{#1} }
\newcommand{\beq}{\begin{equation} \setlength\abovedisplayskip{5pt} 
\setlength\belowdisplayskip{5pt}}
\newcommand{\eeq}{\end{equation}}
\newcommand{\bea}{\begin{eqnarray}}
\newcommand{\eea}{\end{eqnarray}}
\newcommand{\defn}{\stackrel{\triangle}{=}}
\newcommand{\nn}{\nonumber}
\newcommand{\nnl}{\nonumber \\}
\renewcommand{\theequation}{\arabic{equation}}
\renewcommand{\labelenumi}{(\roman{enumi})}

\DeclareMathOperator*{\argmin}{arg\,min}
\DeclareMathOperator*{\argmax}{arg\,max}

\def\ifundefined{\@ifundefined}
\def\bfat{\left[ \begin{array}}
\def\emat{\end{array} \right]}
\def\bfatt{\left\{ \begin{array}}
\def\ematt{\end{array} \right.}
\def\bset{\left\{ \begin{array}}
\def\eset{\end{array} \right\}}
\def\bpar{\left( \begin{array}}
\def\epar{\end{array} \right)}

\graphicspath{{figures/}} % Graphics will be here

\newtheorem{thm}{Theorem}[chapter]
\newtheorem{prop}[thm]{Proposition}
\newtheorem{lem}[thm]{Lemma}
\newtheorem{cor}[thm]{Corollary}

\numberwithin{equation}{chapter}

% BEGIN OF DOCS
\begin{document}
\pagenumbering{roman}


% CONTENTS AND LIST OF FIG AND TABLE PAGES
\tableofcontents

\pagenumbering{arabic}

%#####
% START OF THEORY

\newcommand{\fromeq}[1]{\textit{equation \ref{eq:#1}}}
\newcommand{\fromeqs}[2]{\textit{equations \ref{eq:#1} and \ref{eq:#2}}}

\newcommand{\vecthreeBF}[1]{\vec{\textbf{#1}}}
\newcommand{\vecthree}[1]{\vec{#1}}

\newcommand{\parDeriv}[2]{\frac{\partial #1}{\partial #2}}
\newcommand{\parDerivS}[2]{\frac{\partial^2 #1}{\partial #2^2}}
\newcommand{\derivS}[2]{\frac{d^2 #1}{d#2^2}}

\newcommand{\dotProdBF}[2]{\vecthreeBF{#1} \cdot \vecthreeBF{#2}}
\newcommand{\dotProd}[2]{\vecthree{#1} \cdot \vecthree{#2}}

\newcommand{\crossProdBF}[2]{\vecthreeBF{#1} \times \vecthreeBF{#2}}
\newcommand{\crossProd}[2]{\vecthree{#1} \times \vecthree{#2}}

\chapter{THEORY}

\section{Accelerating Charged Particles}

\subsection{Relation between momentum and acceleration}
In classical mechanics, Newton's second law defines momentum and states 
\begin{equation}
    \vecthreeBF{F} = \frac{\partial \vecthreeBF{p}}{\partial t} = m \frac{\partial \vecthreeBF{v}}{\partial t} = m \vecthreeBF{a}
\end{equation}
In special relativity however, relativistic momentum is defined as $\vecthree{p} = \gamma m_0 \vecthree{v} $ where gamma is the Lorentz Factor;
\begin{equation*}
    \gamma = \frac{1}{\sqrt{1-v^2 / c^2}} = \frac{1}{\sqrt{1-\beta^2}}
\end{equation*}
Considering these two statements, we can find the relation between momentum and acceleration as follows
\begin{eqnarray*}
    \parDeriv{\vecthree{p}}{t} &=& m_0 \parDeriv{(\gamma \vecthree{v})}{t} = m_0 \{  \parDeriv{\gamma}{t}\vecthree{v} + \gamma \parDeriv{\vecthree{v}}{t}  \} \\
    \parDeriv{\gamma}{t} &=& \gamma^3 \dotProd{\beta}{\parDeriv{\beta}{t}} = \frac{\gamma^3}{c} \dotProd{\beta}{a} \\
    \parDeriv{\vecthree{p}}{t}  &=& m_0 \{   \frac{\gamma^3}{c} (\dotProd{\beta}{a} )\vecthree{v} + \gamma \parDeriv{\vecthree{v}}{t}  \} 
\end{eqnarray*}
\begin{equation} \label{eq:Frel}
    \vecthreeBF{F}  = \gamma m_0 \{ \vecthreeBF{a} + \gamma^2(\dotProd{\beta}{\textbf{a}})\vecthree{\beta} \}
\end{equation}
It is clear that acceleration is not necessarily parallel to the force. To start seperating the parallel and perpendicular components relative to $\vecthree{\beta}$, we can find $\vecthreeBF{a}_{||}$ and  $\vecthreeBF{F}_{||}$;
\begin{equation} \label{eq:a_F_parallels}
    \begin{aligned}
        \vecthreeBF{a}_{||} = \frac{(\dotProd{a}{\beta})}{\beta^2}\beta 
    \end{aligned}
    \qquad \qquad
    \begin{aligned}
        \vecthreeBF{F}_{||} = \frac{(\dotProd{F}{\beta})}{\beta^2}\beta 
    \end{aligned}
\end{equation}
\begin{eqnarray*}
    \dotProd{\textbf{F}}{\beta} &=& \gamma m_0 \{ \dotProd{\textbf{a}}{\beta} + \gamma^2(\dotProd{\beta}{\textbf{a}})\beta^2 \} \\
                                &=& \gamma m_0 (\dotProd{\textbf{a}}{\beta}) \{ \gamma^2\beta^2  + 1\} 
\end{eqnarray*}
Using $\gamma^2 \beta^2 + 1 = \gamma^2 $ we have,
\begin{equation*}
    \dotProd{\textbf{F}}{\beta} = m_0 \gamma^3 (\dotProd{\textbf{a}}{\beta} )
\end{equation*}
Inserting this
\begin{eqnarray}
    \vecthreeBF{F}_{||} &=& \frac{(\dotProd{\textbf{F}}{\beta})}{\beta^2} \vec{\beta} \nonumber \\
                        &=& m_0 \gamma^3 \frac{(\dotProd{\textbf{a}}{\beta} )}{\beta^2} \vec{\beta} \nonumber \\
                        &=& m_0 \gamma^3 \vecthreeBF{a}_{||} \label{eq:Frel_parallel}
\end{eqnarray}
Therefore from \fromeqs{Frel}{Frel_parallel}
\begin{eqnarray*}
    \vecthreeBF{F}  &=& m_0 \gamma^3 \vecthreeBF{a}_{||} \beta^2 + m_0 \gamma \vecthreeBF{a} \\
                    &=& m_0 \gamma^3 \vecthreeBF{a}_{||} \beta^2 + m_0 \gamma \{ \vecthreeBF{a}_{||} +  \vecthreeBF{a}_{\perp} \} \\
                    &=& m_0 \vecthreeBF{a}_{||} \gamma \{ \gamma^2\beta^2 + 1 \} + m_0 \gamma \vecthreeBF{a}_{\perp} \\
                    &=& m_0 \vecthreeBF{a}_{||} \gamma^3 + m_0 \gamma \vecthreeBF{a}_{\perp} \\
                    &=& \vecthreeBF{F}_{||} + m_0 \gamma \vecthreeBF{a}_{\perp}
\end{eqnarray*}
\begin{equation} \label{eq:Frel_para_and_perp}
    \begin{aligned}
        \vecthreeBF{F}_{||} = \gamma^3  m_0 \vecthreeBF{a}_{||}
    \end{aligned}
    \qquad \qquad
    \begin{aligned}
        \vecthreeBF{F}_{\perp} = \gamma  m_0\vecthreeBF{a}_{\perp}
    \end{aligned}
\end{equation}


\subsection{Lorentz Force}
Force acting on a charged particle moving in electromagnetic fields is called Lorentz Force and is given by the formula
\newline
\begin{equation} \label{eq:lf}
    \frac{\partial \vecthree{p}}{\partial t} = \vecthree{F}_L=q(\vecthree{E}+ \vecthree{v} \times \vecthree{B})
\end{equation}
\newline
where the $q$ is the charge and $\vecthree{v}$ is the velocity of the particle. 

\subsection{Relativistic Lorentz Force}
Similar to non-relativistic version, relativistic Lorentz Force is given by the following 4-vector equality
\begin{equation}
    \frac{\partial p^{\mu}}{\partial \tau} = q F^{\mu \nu} u_{\nu}
\end{equation}
Where $ \partial \tau = \partial t / \gamma $,
\begin{equation}
    \begin{aligned}
        p^{\mu} = 
        \begin{bmatrix}
            W/c \\
            p_x         \\
            p_y \\
            p_z
        \end{bmatrix}    
    \end{aligned}
    \qquad\qquad
    \begin{aligned}
        F^{\mu\nu} = 
        \begin{bmatrix}
                0       & -E_x/c   & -E_y/c    & -E_z/c \\
                E_x/c   &   0      & -B_z      & B_y     \\
                E_y/c   &   B_z    &  0        & -B_x     \\
                E_z/c   &   -B_y   & B_x       & 0   
        \end{bmatrix} 
    \end{aligned}
    \qquad\qquad
    \begin{aligned}
        u_{\nu} = \gamma
        \begin{bmatrix}
                 c \\
                -v_x \\
                -v_y \\
                -v_z \\
        \end{bmatrix} 
    \end{aligned}
\end{equation}
Where $W$ is the energy of the particle and $\gamma = 1 / \sqrt{1 - v^2 / c^2}$ is the lorentz factor.
\newline

For $ \mu = 0 $, we have the time component of the equation;
\begin{eqnarray}
    \frac{\gamma}{c}\frac{\partial W}{\partial t} &=& \frac{q \gamma \vecthree{E}\cdot \vecthree{v}}{c} = \frac{q \gamma}{c}  \frac{\vecthree{E}  \cdot \partial \vecthree{r}}{\partial t} \\
    \frac{\partial W}{\partial t} &=& q \frac{\vecthree{E}\cdot \partial \vecthree{r}} {\partial t}
\end{eqnarray}
This is the definition of work done by an electric field. For $ \mu = 1,2,3 $, we have the spacial components;
\begin{equation*}
    \frac{\partial \vecthree{p}}{\partial \tau} = \gamma \frac{\partial \vecthree{p}}{\partial t} = q \gamma (\vecthree{E} + \vecthree{v} \times \vecthree{B})
\end{equation*}
Which simplifies to non-relativistic Lorentz Force in \fromeq{lf}.


\subsection{Acceleration caused by lorentz force}
Due to the nature of the cross product, lorentz force caused by a magnetic field is always perpendicular to the velocity of the particle. 
Therefore the acceleration of the magnetic field is straightforward
\begin{equation*}
    \vecthreeBF{F}_B = \vecthreeBF{F}_{\perp} = \gamma  m_0\vecthreeBF{a}_{\perp} = \gamma  m_0\vecthreeBF{a}_{B}
\end{equation*}
The same thing cannot be said about electric field however. 
It can create force in any direction with respect to velocity. Therefore, we have the following equality;
\begin{equation} \label{eq:para_and_perp_acc_of_lorentz_force}
    \begin{aligned}
        \vecthreeBF{a}_{||} = \frac{q }{\gamma^3 m_0} \vecthreeBF{E}_{||}
    \end{aligned}
    \qquad \qquad
    \begin{aligned}
        \vecthreeBF{a}_{\perp} = \frac{q}{\gamma m_0} \{ \vecthreeBF{E}_{\perp} + \crossProdBF{v}{B}   \}
    \end{aligned}    
\end{equation}
Acceleration due to electric field can be simplified as;
\begin{eqnarray*}
    \vecthreeBF{a}_{\{B=0\}} = \vecthreeBF{a}_{E}   &=&  \vecthreeBF{a}_{||} + \vecthreeBF{a}_{\perp \{B=0\}} \\
                                &=& \frac{q}{m_0 \gamma} \{ \frac{\vecthreeBF{E}_{||}}{\gamma^2} + \vecthreeBF{E}_{\perp} \} \\
                                &=& \frac{q}{m_0 \gamma} \{ \{1 - \beta^2\}\vecthreeBF{E}_{||} + \vecthreeBF{E}_{\perp} \} \\
                                &=& \frac{q}{m_0 \gamma} \{ \vecthreeBF{E}_{||} + \vecthreeBF{E}_{\perp}-\beta^2\vecthreeBF{E}_{||} \} \\
                                &=& \frac{q}{m_0 \gamma} \{ \vecthreeBF{E} - \beta^2\vecthreeBF{E}_{||} \} \\
\end{eqnarray*}
Using the fact that $\vecthreeBF{E}_{||} = \vec{\beta}(\dotProd{\textbf{E}}{\beta})/\beta^2 $, we finally have
\begin{equation} \label{eq:acc_of_E_and_B}
    \begin{aligned}
        \vecthreeBF{a}_{E} = \frac{q}{m_0 \gamma} \{ \vecthreeBF{E} - \vecthreeBF{v}\frac{(\dotProdBF{E}{v})}{c^2} \}
    \end{aligned}
    \qquad \qquad
    \begin{aligned}
        \vecthreeBF{a}_{B} = \frac{q}{\gamma m_0} ( \crossProdBF{v}{B} )
    \end{aligned}
\end{equation}


\section{Practical}

\subsection{Accelerators}
Particle accelerators are sophisticated scientific instruments designed to accelerate charged particles, such as electrons, protons, or ions, to high speeds and energies. 
These accelerators play a crucial role in advancing our understanding of the fundamental properties of matter and the universe. 
They are widely used in various fields of research, including particle physics, nuclear physics, materials science, and medicine.

At their core, particle accelerators utilize electromagnetic fields to impart energy to particles and control their trajectories. 
These fields are generated by intricate arrangements of magnets and RF (radiofrequency) cavities within the accelerator structure. 
By precisely controlling these fields, accelerators can propel particles to speeds close to the speed of light, enabling them to acquire high energies.

Accelerators can be categorized into two main types: linear accelerators (linacs) and circular accelerators. 
Linacs accelerate particles in a straight line, while circular accelerators use magnetic fields to bend the particle trajectory into a circular path. 

The acceleration process in accelerators involves multiple stages. Initially, particles are injected into the accelerator at a relatively low energy. 
As they progress through the accelerator, they are subjected to alternating electric fields that accelerate them, while magnetic fields guide their trajectories.
Focusing elements, such as magnetic lenses or quadrupole magnets, ensure the particles remain tightly controlled.

As particles gain energy in the accelerator, they approach relativistic speeds, where relativistic effects become significant. 
Special relativity governs the increase in mass and energy of the particles as they approach the speed of light, providing insights into the behavior of matter at high energies.

Particle accelerators are essential tools for scientific research. They enable scientists to probe the fundamental constituents of matter, study particle interactions, and explore the laws of physics. 
Accelerators have been instrumental in discovering and characterizing fundamental particles, such as quarks, leptons, and the Higgs boson. 
They also facilitate the production of high-intensity beams for applications in material science, radiotherapy, and industrial processes like particle irradiation and sterilization.

In summary, particle accelerators are complex scientific instruments that use electromagnetic fields to accelerate charged particles to high speeds and energies. 
They enable scientists to investigate the fundamental nature of matter, unravel the mysteries of the universe, and apply their findings to various fields of research and practical applications.

% TODO : ADD CIRCULAR AND LINEAR ACCELERATOR PICTURES HERE

\subsubsection{Key Concepts}

\begin{itemize}
    \item Phase Lag
    
    This is the asdasd

    \item XD
    \item AW
    
    
\end{itemize}

\subsubsection{Acceleration Cavities}
Radiofrequency (RF) cavities, also known as accelerating cavities or resonant cavities, are key components in particle accelerators. 
These cavities generate strong electromagnetic fields at specific frequencies to accelerate charged particles through clever engineering.

RF cavities are typically hollow metallic structures made of high-conductivity materials such as copper or niobium. 
They are designed to resonate at a specific frequency, which is determined by the size and shape of the cavity. 
The cavity is often cylindrical or spherical in shape, and its inner surface is polished to minimize energy losses through resistive heating.
The RF cavity is designed to be resonant, meaning that it naturally amplifies the electric fields at its resonant frequency. 
The resonant frequency is determined by the cavity's dimensions and the speed of light in the cavity material.
To achieve efficient energy transfer to the particles, the RF cavity is driven by an external RF power source operating at the resonant frequency. 
The power source supplies radiofrequency energy to the cavity, which causes the electric fields inside the cavity to oscillate at the desired frequency. 
These oscillating fields then transfer energy to the passing particles, increasing their kinetic energy by pushing and pulling on the charged particles as they pass through the cavity. 

In addition to accelerating the particles, RF cavities are often designed to provide focusing forces. 
By carefully shaping the cavity and adjusting the electromagnetic fields, the particles can experience focusing effects as they pass through the cavity. 
This helps to maintain a tight and controlled beam. To ensure efficient acceleration, it is essential to maintain phase stability. 
This means that the particles should experience the strongest electric fields at the correct time during their passage through the cavity. 
Precise timing and synchronization of the RF power source with the particle beam are crucial to achieve phase stability and maximize energy transfer.

% TODO : ADD RF CAVITY PICTURE HERE

\subsubsection{Steerer Magnets}
Steerer magnets, also known as dipole magnets or bending magnets, are fundamental components used in particle accelerators to control the trajectory of charged particles. 
They utilize the Ampere's Law to exert a magnetic field that interacts with the charged particles in the accelerator. 
According to the Lorentz Force Law ( $\textit{Section \ref{sec:LorentzForce}}$ ), when a charged particle moves through a magnetic field, it experiences a force perpendicular to both its velocity 
vector and the magnetic field direction. This force causes the particle's trajectory to curve, resulting in a bending effect.

% TODO : ADD STEERER MAGNET PICTURE HERE

\subsection{Rhodotron Accelerator}



\section{Numerical Integration Methods}

\subsection{Leapfrog} \label{sec:leapfrog}
The Leapfrog method is a numerical method commonly used to solve ordinary differential equations (ODEs) that involve second-order time derivatives. Such an ODE is shown below:

\begin{equation}
    \ddot{x} = \derivS{x}{t} = f(x) \label{eq:second-order-ODE}
\end{equation}
\newline
The Leapfrog method is a variant of the finite difference method, and it approximates the solution of an ODE by discretizing both time and space. The method gets its name from the way it calculates the values of the solution at each time step, which resembles a "leapfrogging" motion. 
It is a simple and efficient algorithm that is often used in simulations of physical systems, such as celestial mechanics or molecular dynamics.
The idea is straight forward; in the time interval $\Delta t$, 
\begin{eqnarray}
    a(t_0) &=& f(x_0) \nonumber \\
    x(t_0 + \Delta t) &=& x(t_0) + v(t_0)\Delta t + a(t_0)\frac{\Delta t^2}{2} \label{eq:leapfrog_sync_x}\\
    v(t_0 + \Delta t) &=& v(t_0) + \{ a(t_0) + a(t_0 + \Delta t)\}\frac{\Delta t}{2}  \label{eq:leapfrog_sync_v}
\end{eqnarray}

For more stability, this version can be rearranged to what is called 'kick-drift-kick' form,
\begin{eqnarray} \label{eq:leapfrog}
    v(t_0 + \Delta t/2) &=& v(t_0) +  a(t_0)\frac{\Delta t}{2} \nonumber \\
    x(t_0 + \Delta t) &=& x(t_0) + v(t_0 + \Delta t/2)\Delta t \\
    v(t_0 + \Delta t) &=& v(t_0 + \Delta t/2) + a(t_0 + \Delta t)\frac{\Delta t}{2}  \nonumber
\end{eqnarray}

This version provides more time resolution to our calculation; however, it increases the number of calculations needed by about $50\%$.

\subsection{Runge Kutta}

The Runge-Kutta method is another numerical method used to solve ordinary differential equations (ODEs) numerically (see $\fromeq{second-order-ODE}$). It's named after the German mathematicians Carl Runge and Martin Wilhelm Kutta.

The basic idea behind the Runge-Kutta method is to approximate the solution of an ODE by taking small steps and using a weighted average of function evaluations at different points within each step. 
This weighted average improves the accuracy of the approximation compared to simpler methods like Leapfrog (see $\textit{section \ref{sec:leapfrog}}$).
% TODO: ADD GENERAL FORM OF RK HERE
\newline
The most commonly used version of the Runge-Kutta method is the fourth-order Runge-Kutta method, also known as RK4. The RK4 method involves four function evaluations per step and has an error term that is proportional to the step size raised to the fifth power.

% TODO: ADD RK4 HERE  



\section{Simulation}


A simulation software is a computer program or tool that enables the creation and execution of 
simulations to model and analyze real-world systems or processes. 
It allows users to replicate the behavior, interactions, and outcomes of the system or process under study, 
providing insights and predictions that can be valuable for decision-making, optimization, or understanding complex phenomena.

Simulation software provides a virtual environment where users can define the parameters,
variables, and rules of the system being simulated. The software then uses mathematical models, 
algorithms, and computational techniques to simulate the behavior of the system over time.

\subsection{}



% END OF THEORY
%#####
\end{document}
